\documentclass{article}

\title{Q Vector Computation}
\author{Carlos Perez Lara\and Jaehyeon Do \and Veronica Canoa Roman}
\date{Last updated: \today}

\begin{document}
\maketitle

\section{Definitions}
%===================Q Vector
The {\bf Q vector} is an event-by-event observable defined as
\begin{equation}
\label{eq.q}
Q_n = \sum_{i} \omega_i e^{i n\varphi_i} = \vert Q_n \vert e^{i n\Psi_n}
\end{equation}
where he sum goes over all particles in the sample and $\varphi_i$ is the laboratory azimuthal angle of the particles emerging from the collision.
When the sample is composed of reconstructed particles, the weight $\omega_i$  is sometimes different than 1 in order to enhance the contribution of particles with the highest contribution to the flow, on the other hand, for some detector configurations (like BBC, FVTX and MPCEX) the weight $\omega_i$ is the relative energy flow.

The {\bf symmetry angle} $\Psi_n$ is the angle associated to the Q vector and is defined in equation \ref{eq.q}.

\section{Correcting Q$_n$ against biases}
The Q vector can be computed by any detector that has azimuthal granularity (trackers, hodoscopes, calorimeters, etc).
For a proper unbiased estimator, one need to correct by detector non-uniform acceptance and efficiency.
The corrections used here are computed from the data itself and consist of several steps (more information on these methods can be found in \cite{PhysRevC.77.034904,PhysRevC.56.3254}):

{\bf 1. Recentering the Q centroid.}
\begin{equation}
Q^\mathrm{I}_{nx} = Q_{nx}(1-\langle \cos{n\varphi}\rangle), \qquad
Q^\mathrm{I}_{ny} = Q_{ny}(1-\langle \sin{n\varphi}\rangle)
\end{equation}

{\bf 2. Twist of the Q vector.}
\begin{equation}
Q^\mathrm{II}_{nx} = \frac{Q^\mathrm{I}_{nx}-\lambda^{s-}_{2n} Q^\mathrm{I}_{ny}}{1-\lambda^{s-}_{2n}\lambda^{s+}_{2n}}, \,
Q^\mathrm{II}_{ny} = \frac{Q^\mathrm{I}_{ny}-\lambda^{s+}_{2n} Q^\mathrm{I}_{nx}}{1-\lambda^{s-}_{2n}\lambda^{s+}_{2n}}, \,
\lambda^{s\pm}_{2n}=\frac{\langle\sin{2n\varphi}\rangle}{1\pm\langle\cos{2n\varphi}\rangle}
\end{equation}

{\bf 3. Scaling the Q vector.}
\begin{equation}
Q^\mathrm{III}_{nx} = \frac{Q^\mathrm{II}_{nx}}{1+\langle\cos{2n\varphi}\rangle}, \qquad
Q^\mathrm{III}_{ny} = \frac{Q^\mathrm{II}_{ny}}{1-\langle\cos{2n\varphi}\rangle}
\end{equation}

{\bf 4. Flattening the symmetry angle.}
\begin{equation}
\Psi^\mathrm{IV}_{n} = \Psi^\mathrm{III}_{n} + \sum_{m}\frac{2}{m}( -\langle\sin{m\Psi^\mathrm{III}_n}\rangle\cos{m\Psi^\mathrm{III}_n} +\langle\cos{m\Psi^\mathrm{III}_n}\rangle\sin{m\Psi^\mathrm{III}_n})
\end{equation}

Each of these effects were corrected sequentially, which requires several passes over data. Notice that if there were not any bias, each of the averages "$\langle\rangle$" would be zero and thus $Q^\mathrm{IV}=Q^\mathrm{III}=Q^\mathrm{II}=Q^\mathrm{I}=Q$ we recover the original Q.

\section{BBC Q$_n$ vectors}
\begin{figure}
\label{fig.bbcgeo}
\caption{Geometrical sketch of the BBC south arm sub-detectors}
\end{figure}
The BBC detector consist of two arrays of detectors, each covering either $-2.9<\eta<-2.1$ (north) or $2.1<\eta<2.9$ (south).
Each arm consists of 64 detectors. Figure \ref{fig.bbcgeo} depicts the geometrical arrange in the x-y plane for the south arm.
Following equation \ref{eq.q} the BBC-based Q vectors can be constructed expanding the sum over the 64 detectors and using the center of each detector for the determination of $\varphi_i$ and the calibrated ADC signal of each detector as weight $\omega_i$.

{\bf Q$_n$ accuracy.}
The BBC accuracy of the measured the Q vector can be quantified by measuring the resolution of its $\Psi_n$ counterpart.
There are several methods to do that. In this analysis we rely on measuring the resolution of the detector by using subevents within the same detector. This is done in order to prevent using other detector with different detector azimuthal non-uniformities or, worse yet, the coupling of longitudinal expansion effects.
For the BBC this can be done since each arm side can be separated into two independent sub-detector each equally sensitive to the event multiplicity and its flow and having approximately the same geometrical configuration.
The resolution can then be computed using the methods in \cite{PhysRevC.77.034904}. Namely for the strongest contribution:
\begin{equation}
\langle\cos{n(\Psi_n-\Psi)}\rangle = \frac{\sqrt{\pi}}{2\sqrt{2}}\chi_n e^{-\chi_n^2/4}\lbrace I_0(\chi_n^2/4)-I_1(\chi_n^2/4)\rbrace
\end{equation}

\section{Run16 dAu 200 GeV}

\bibliography{references}{}
\bibliographystyle{plain}
\end{document}

