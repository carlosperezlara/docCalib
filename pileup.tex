\documentclass{article}
\usepackage{graphicx}
\title{PileUp Rejection Criteria based on BBC}
\author{Carlos Perez Lara\and Jaehyeon Do \and Veronica Canoa Roman}
\date{Last updated: \today}

\begin{document}
\maketitle

\section{The BBC detector}
\section{The BBC timing coincidence}
\begin{figure}
\includegraphics[width=0.48\textwidth]{fig_pileup/dAu200MB_rmsSouth_455547}
\includegraphics[width=0.48\textwidth]{fig_pileup/dAu200MB_rmsNorth_455547}
\label{fig.dau200mb.bbctimetime}
\caption{Raw distribution of time coincidence for BBC in dAu collisions from 0 to 5 \% centrality at 200 GeV. Each pannel represents situation for one run.}
\end{figure}


\section{Run16 dAu 200 GeV}
\begin{figure}
\includegraphics[width=\textwidth]{fig_pileup/dAu200MB_TimingBBC_TimeTimeBC}\\
\label{fig.dau200mb.bbctimetime}
\caption{Raw distribution of time coincidence for BBC in dAu collisions from 0 to 5 \% centrality at 200 GeV. Each pannel represents situation for one run.}
\end{figure}

\begin{figure}
\includegraphics[width=\textwidth]{fig_pileup/dAu200MB_TimingBBC_RMS}
\includegraphics[width=\textwidth]{fig_pileup/dAu200MB_TimingBBC_Time}
\label{fig.dau200mb.bbctimetime}
\caption{Performance of BBC timing cut  in dAu collisions from 0 to 5 \% centrality at 200 GeV. Each pannel represents situation for one run. The bottom plot corresponds to the difference between the north and south estimation of time event-by-event , while the top plot corresponds to the RMS within the south arm. The blue distributions are for the variables before the coincidence cut. The red distributions are after the cut.}
\end{figure}

\begin{figure}
\includegraphics[width=\textwidth]{fig_pileup/dAu200MB_TimingBBC_BBCM}\\
\includegraphics[width=\textwidth]{fig_pileup/dAu200MB_TimingBBC_RATEBBCM}
\label{fig.dau200mb.bbcrate}
\caption{Performance of the pileup rejection cut in dAu 0-5\% centrality class. The top plot correspond tot he the total BBCsouth signal as a function of time (the analysis was done for each run segment). The bottom plot is the depedence of the BBCsouth signal run by run to the total instantanoeus rate.}
\end{figure}


%\bibliography{references}{}
%\bibliographystyle{plain}
\end{document}

